% ======================================================================
% scrjura-en.tex
% Copyright (c) Markus Kohm, 2011-2024
%
% This file is part of the LaTeX2e KOMA-Script bundle.
%
% This work may be distributed and/or modified under the conditions of
% the LaTeX Project Public License, version 1.3c of the license.
% The latest version of this license is in
%   http://www.latex-project.org/lppl.txt
% and version 1.3c or later is part of all distributions of LaTeX 
% version 2005/12/01 or later and of this work.
%
% This work has the LPPL maintenance status "author-maintained".
%
% The Current Maintainer and author of this work is Markus Kohm.
%
% This work consists of all files listed in MANIFEST.md.
% ======================================================================
%
% Chapter about scrjura of the KOMA-Script guide
% Maintained by Markus Kohm
%
% ======================================================================

\KOMAProvidesFile{scrjura-en.tex}%
                 [$Date: 2024-02-13 16:06:28 +0100 (Tue, 13 Feb 2024) $
                  KOMA-Script guide (chapter: scrjura)]

\translator{Markus Kohm}

\chapter{Support for the Law Office with \Package{scrjura}}
\labelbase{scrjura}
\BeginIndexGroup
\BeginIndex{Package}{scrjura}
\BeginIndex{Package}{contract}

Up to and including version~3.41, \KOMAScript{} officially provided the
package \Package{scrjura} to support legal documents. These were mainly
statutes, laws, commentaries on them or, in the broadest sense, contracts of
all kinds. In the course of the restructuring of \KOMAScript{}, the package
was spun off. Since the contract is the central element of the package, it was
given the new name \Package{contract}. Under this name it can not only be
found on CTAN (see \cite{package:contract}). It has also been integrated into
common \TeX{} distributions and can be installed via their package manager.

For compatibility reasons, there will still be a package \Package{scrjura} in
\KOMAScript{} for the time being. However, this is merely a different
packaging of the new \Package{contract}, in which some of the
incompatibilities between the new package and the previous \Package{scrjura}
have been fixed. Therefore, it should generally be possible to continue
editing existing documents based on \Package{scrjura}. For new packages, it is
strongly recommended to switch to \Package{contract} instead. Switching is
also recommended when revising old documents. Please refer to the
\Package{contract} user manual for the changes to be taken into account.

\begin{Declaration}
  \Macro{Clause}\Parameter{options}%
  \Macro{SubClause}\Parameter{options}
\end{Declaration}
The most important difference when using \Package{scrjura} compared to
\Package{contract} is that the \PName{options} for the two instructions
\Macro{Clause} and \Macro{SubClause} within a \Environment{contract}
environment with package \Package{contract} represent an optional argument,
i.e. must be specified in square brackets. In the \Package{scrjura} package,
however, the \PName{options} were always a required argument, i.e. to be
specified in braces. This is still the case with \Package{scrjura}.%
%
\EndIndexGroup
%
\EndIndexGroup

\endinput

%%% Local Variables: 
%%% mode: latex
%%% TeX-master: "scrguide-en.tex"
%%% coding: utf-8
%%% ispell-local-dictionary: "en_GB"
%%% eval: (flyspell-mode 1)
%%% End: 
