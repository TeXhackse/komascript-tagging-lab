% \iffalse meta-comment
%
%% File: scr-taggging-lab.dtx
%
%<*driver>
\documentclass[USenglish]{koma-script-source-doc}
\EnableCrossrefs
\CodelineIndex
\begin{document}
  \DocInput{scr-tagging-lab.dtx}
\end{document}
%</driver>
%
% \fi
%
%    \begin{macrocode}
%<@@=scrlab>
%<*testphase>
\ProvidesExplFile{KOMA-fixes-latex-lab-testphase.ltx}{2024-09-14}{v0.1}{KOMA-fixes testphase}
\RequirePackage{scr-tagging-lab}
%</testphase>
%
%<*package|koma>
%    \end{macrocode}
%
% \section{Implementation}
%    \begin{macrocode}
\ProvidesExplPackage {scr-tagging-lab} {2024-06-30} {0.18}{KOMA-Script tagging experiments}
%    \end{macrocode}
% As this project is constructed to be integrated into KOMA-Script, we define some macros which are also defined by KOMA-Script itself.
% We use file hooks to do save the meaning before loading the class and restore it afterwards.
% This is to simplify moving the changes from this experiment to the official KOMA-Script SVN.
%    \begin{macrocode}
\clist_new:N \g_@@_changed_macros_clist
\clist_gset:Nn  \g_@@_changed_macros_clist {
	scr@dte@tocline,
	sectionlinesformat,
	scr@numberline,
	scr@saved@startsection,
	@seccntformat
}

\cs_new:Nn \_@@_capture_changed_macros: {
	\clist_map_inline:Nn \g_@@_changed_macros_clist {
		\cs_set_eq:cc {_@@_##1} {##1}
		\cs_undefine:c {##1}
	}
	\def\@seccntformat{}
}

\cs_new:Nn \_@@_restore_changed_macros: {
	\clist_map_inline:Nn \g_@@_changed_macros_clist {
		\cs_set_eq:cc {##1}{_@@_##1}
	}
	\let\@sect\scr@lab@sect
	\let\@xsect\scr@lab@xsect
}

\hook_gput_code:nnn {class/before}{scr-tagging-lab-catch-meanings}  {
	\_@@_capture_changed_macros:
}
\hook_gput_code:nnn {class/after}{scr-tagging-lab-catch-meanings}  {
	\_@@_restore_changed_macros:
}

% save and restore \TOCLineLeaderFill before and after tocbasic
\hook_gput_code:nnn {package/tocbasic/before} {toclineleaderfill}{
	\cs_set_eq:cc{_@@_TOCLineLeaderFill} {\string\TOCLineLeaderFill}
	\cs_undefine:c {TOCLineLeaderFill}
}

\hook_gput_code:nnn {package/tocbasic/after}{toclineleaderfill}{
	\cs_set_eq:cc {\string\TOCLineLeaderFill}{_@@_TOCLineLeaderFill}
}

\iffalse
\providecommand{\@startsection}{\scr@saved@startsection
\newcommand{\scr@saved@footnotetext}{}}
\scr@saved@footnotemark
\newcommand*{\scr@saved@startsection}[6]{%
\fi
%    \end{macrocode}
% Insted of redefining \cs{scr@startsection} it's possible to use hooks.
% Be aware that using the generic hook needs a check to limit the action to exclude chapter/part.
%    \begin{macrocode}
\cs_new:Nn \__scr_scr_start_aux:n {
	\clist_if_in:nnF { chapter, part } {#1} {
		\tag_tool:n {sec-start=#1}
	}
}

\hook_gput_code:nnn {class/after}{scr-tagging-lab-scrstart}  {
	\AddtoDoHook{heading/branch/star}{\__scr_scr_start_aux:n}
	\AddtoDoHook{heading/branch/nostar}{\__scr_scr_start_aux:n}
}

%    \end{macrocode}
% \begin{macro}{\@sect}
% As the content of scrkernel-sections.dtx is overwriting the orignal definition of \cs{@sect} and some other macros.
% This change has to be adjusted to match the kernel changes of the testphase.
% The Following code has been taken from scrkernel-sections.dtx by Markus Kohm.
% Changes have been marked
%    \begin{macrocode}
%</package|koma>
%    \end{macrocode}
% Because the KOMA hooks don't have additional arguments it's necessary to use hooks with arguments on the surrounding macro to save the title.
% Maybe it's easier to pass those with changing the hooks, but this needs to be discussed.
%    \begin{macrocode}
%<*package|heading-hooks>

\tl_new:N \__scr_current_heading_tl
%\hook_new:n {scr/heading/endgroup/chapter}%TODO
\hook_gput_code_with_args:nnn {cmd/scr@@startchapter/before} {tagging-save-heading} {
	\tl_set:Nn \__scr_current_heading_tl {#2}
}
\hook_gput_code_with_args:nnn {cmd/scr@@startpart/before} {tagging-save-heading} {
	\tl_set:Nn \__scr_current_heading_tl {#2}
}

\hook_gput_code:nnn {class/after} {scr-tagging-lab-sections}  {
	\AddtoDoHook{heading/begingroup}{\tagtool{para-flattened=true}\use_none:n}
	\AddtoDoHook{heading/begingroup/chapter}{
		\exp_args:No \tag_tool:n { sec-start-chapter=  \__scr_current_heading_tl  }
		\Ifnumbered{chapter} \relax {\MakeLinkTarget[chapter]}
	}
	\AddtoDoHook{heading/endgroup/chapter}{
	   \tag_tool:n {sec-stop-chapter}
	}
	\AddtoDoHook{heading/begingroup/part}{
		\exp_args:No \tag_tool:n { sec-start-part=  \__scr_current_heading_tl  }
		\IfUseNumber\relax{\MakeLinkTarget[part]{}}
	}
	\AddtoDoHook{heading/begingroup/part}{
		\exp_args:No \tag_tool:n { sec-start-part=  \__scr_current_heading_tl  }
		\Ifnumbered{part} \relax {\MakeLinkTarget[part]}
	}
	\AddtoDoHook{heading/endgroup/part}{
	   \tag_tool:n {sec-stop-part}
	}
}
%    \end{macrocode}
% \begin{macro}{\scrlab@section@target}
% \changes{tagging}{2024/09/15}{Adapt our own \cs{@hyp@section@target@nnn}  as  \cs{\scrlab@section@target}}
% tagging: Hyperref has some internal command \cs{@hyp@section@target@nnn} to move the anchor in case the indent in front of the section heading is negative.
% We copy this command to adjust the anchor in the same way.
% The definition is taken from the \pkg{latex-lab-sec}.dtx.
%    \begin{macrocode}
%% This definition was adapted from
%% latex-lab-sec.dtx
%% Copyright 2021-2023 LaTeX Project
\cs_new_protected:Npn \scrlab@section@target #1 #2 #3 %#1 optarg #2 name/counter, #3 indent
  {
    \makebox[0pt][l]
     {
       \skip_set:Nn \@tempskipa {#3}
       \dim_compare:nNnF {\@tempskipa}<{0pt}{\kern-\@tempskipa}
       \MakeLinkTarget#1{#2}
     }
  }
%    \end{macrocode}
% \end{macro}^^A \scrlab@section@target
%</package|heading-hooks>
%    \end{macrocode}
