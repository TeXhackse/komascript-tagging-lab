% \iffalse^^A meta-comment
% ======================================================================
% scrlfile-hook.dtx
% Copyright (c) Markus Kohm, 2021-2022
%
% This file is part of the work `scrlfile' which is part of the LaTeX2e
% KOMA-Script bundle.
%
% This work may be distributed and/or modified under the conditions of
% the LaTeX Project Public License, version 1.3c of the license.
% The latest version of this license is in
%   http://www.latex-project.org/lppl.txt
% and version 1.3c or later is part of all distributions of LaTeX 
% version 2005/12/01 or later and of this work.
%
% This work has the LPPL maintenance status "author-maintained".
%
% The Current Maintainer and author of this work is Markus Kohm.
%
% This work consists of all files listed in MANIFEST.md.
% ======================================================================
%
%%% From File: $Id: scrlfile-hook.dtx$
%<*dtx>
\ifx\ProvidesFile\undefined\def\ProvidesFile#1[#2]{}\fi
\begingroup
  \def\filedate$#1: #2-#3-#4 #5${\gdef\filedate{#2/#3/#4}}
  \filedate$Date: 2022-06-05 12:38:02 +0200 (Sun, 05 Jun 2022) $
  \def\filerevision$#1: #2 ${\gdef\filerevision{r#2}}
  \filerevision$Revision: 3874 $
  \edef\reserved@a{%
    \noexpand\endgroup
    \noexpand\ProvidesFile{scrlfile-hook.dtx}%
                          [\filedate\space\filerevision\space
                           KOMA-Script package source
  }%
\reserved@a
%</dtx>
%<package&current>\ProvidesPackage{scrlfile-hook}[%
%<package&3.34>\ProvidesPackage{scrlfile-hook-3.34}[%
%!KOMAScriptVersion
%<package>  package
  (using LaTeX hooks)]
%<*dtx>
\ifx\documentclass\undefined
  \input scrdocstrip.tex
  \@@input scrkernel-version.dtx
  \@@input scrstrip.inc
  \KOMAdefVariable{COPYRIGHTFROM}{2002}
  \generate{\usepreamble\defaultpreamble
    \file{scrlfile-hook.sty}{%
      \from{scrlfile-hook.dtx}{package,current}%
      \from{scrlogo.dtx}{logo}%
    }%
    \file{scrlfile-hook-3.34.sty}{%
      \from{scrlfile-hook.dtx}{package,3.34}%
      \from{scrlogo.dtx}{logo}%
    }%
  }%
  \@@input scrstrop.inc
\else
  \let\endbatchfile\relax
\fi
\endbatchfile
\documentclass[USenglish]{l3doc}% Using class for documentation of expl3 style packages.
\usepackage{babel}
\expandafter\let\expandafter\nodriver\csname iffalse\endcsname
\expandafter\let\expandafter\endnodriver\csname fi\endcsname
\makeatletter% \iffalse^^A meta-comment
% ======================================================================
% scrlogo.dtx
% Copyright (c) Markus Kohm, 2002-2023
%
% This work may be distributed and/or modified under the conditions of
% the LaTeX Project Public License, version 1.3c of the license.
% The latest version of this license is in
%   http://www.latex-project.org/lppl.txt
% and version 1.3c or later is part of all distributions of LaTeX
% version 2005/12/01 or later and of this work.
%
% This work has the LPPL maintenance status "author-maintained".
%
% The Current Maintainer and author of this work is Markus Kohm.
%
% This work consists of the file `scrlogo.dtx' at least.
% ======================================================================
%
%%% From File: $Id: scrlogo.dtx 4032 2023-04-17 09:45:11Z kohm $
%<*dtx>
\ifx\ProvidesFile\undefined\def\ProvidesFile#1[#2]{}\fi
\begingroup
  \def\filedate$#1: #2-#3-#4 #5${\gdef\filedate{#2/#3/#4}}
  \filedate$Date: 2023-04-17 11:45:11 +0200 (Mon, 17 Apr 2023) $
  \def\filerevision$#1: #2 ${\gdef\filerevision{r#2}}
  \filerevision$Revision: 2631 $
  \edef\reserved@a{%
    \noexpand\endgroup
    \noexpand\ProvidesFile{scrlogo.dtx}%
                          [\filedate\space\filerevision\space
                           KOMA-Script package source
  }%
\reserved@a
%</dtx>
%<*dtx|package>
%<package>\ProvidesPackage{scrlogo}[%
%!KOMAScriptVersion
%<package>   package
  (logo)]
%</dtx|package>
%<*dtx>
\ifx\documentclass\undefined
  \ifx\ifTopLevel\undefined
    \input scrdocstrip.tex
    \@@input scrkernel-version.dtx
    \@@input scrstrip.inc
    \KOMAdefVariable{COPYRIGHTFROM}{2002}
    \generate{\usepreamble\defaultpreamble
      \file{scrlogo.sty}{\from{scrlogo.dtx}{package,logo}}%
    }
    \@@input scrstrop.inc
  \fi
\else
  \let\endbatchfile\relax
\fi
\endbatchfile
% Make it possible to load scrlogo.dtx directly.
\csname nodriver\endcsname
\expandafter\let\csname endnodriver\endcsname\relax
\documentclass[USenglish]{koma-script-source-doc}
\usepackage{babel}
\setcounter{StandardModuleDepth}{2}
\begin{document}
  \DocInput{scrlogo.dtx}
\end{document}
\endnodriver
%</dtx>
% \fi^^A meta-comment
%
% \changes{v3.36}{2022/02/02}{switch over from \cls*{scrdoc} to
%   \cls*{koma-script-source-doc}}
% \changes{v3.36}{2022/02/02}{whole implementation documentation in English}
% \changes{v3.40}{2023/04/17}{guide names changed}
%
% \GetFileInfo{scrlogo.dtx}
% \title{\href{https://komascript.de}{\KOMAScript} \partname\ \file{\filename}}
% \author{\href{mailto:komascript@gmx.info}{Markus Kohm}}
% \date{Revision \fileversion{} of \filedate}
% \maketitle
% \begin{abstract}
%   This part of \KOMAScript{} provides not only the \cs{KOMAScript} command
%   for all \KOMAScript{} classes and packages, but also a standalone package
%   to provide the logo.
% \end{abstract}
% \tableofcontents
%
% \section{User Manual}
%
% You can find the user manual of \pkg*{scrlogo} in the \KOMAScript{} manual,
% either the German \file{scrguide-de.pdf} or the English
% \file{scrguide-en.pdf}.
%
% \MaybeStop{\PrintIndex}
%
% \section{Implementation of the \KOMAScript{} Logo}
%
% \begin{command}{\KOMAScript}
% \changes{v3.36}{2022/02/02}{respect \pkg{hyperref}}
% \changes{v3.36}{2022/02/02}{other packages and classes load \pkg*{scrlogo}
%   instead of defining the command itself}
% From version~3.36, the letter spacing of the upper case letters is not used
% in PDF context of package \pkg{hyperref}.
%    \begin{macrocode}
%<*logo>
%<forcedefine>\IfFileExists{scrlogo.sty}{%
%<!package>\RequirePackage{scrlogo}
%<forcedefine>}{}%
%<*(package|forcedefine)&logo>
\@ifundefined{KOMAScript}{%
  \DeclareRobustCommand*{\KOMAScript}{\textsf{K\kern.05em O\kern.05em%
      M\kern.05em A\kern.1em-\kern.1em Script}}%
  \@ifundefined{pdfstringdefPreHook}{%
    \def
  }{%
    \g@addto@macro
  }%
  \pdfstringdefPreHook{%
    \def\KOMAScript{KOMA-Script}%
    \expandafter\def\csname KOMAScript \endcsname{KOMA-Script}%
  }%
}{}
%</(package|forcedefine)&logo>
%</logo>
%    \end{macrocode}
% \end{command}
%
% \Finale
% \PrintChanges
%
\endinput
% Local Variables:
% mode: doctex
% ispell-local-dictionary: "en_US"
% eval: (flyspell-mode 1)
% TeX-master: t
% TeX-engine: luatex-dev
% eval: (setcar (or (cl-member "Index" (setq-local TeX-command-list (copy-alist TeX-command-list)) :key #'car :test #'string-equal) (setq-local TeX-command-list (cons nil TeX-command-list))) '("Index" "mkindex %s" TeX-run-index nil t :help "makeindex for dtx"))
% End:
\makeatother
\CodelineIndex

\begin{document}
  \DocInput{scrlfile-hook.dtx}
\end{document}
%</dtx>
% \fi^^A meta-comment
%
% \changes{v3.32}{2020/08/25}{new (sub-)package}
% \changes{v3.36}{2022/02/03}{use new documentation style}
% \changes{v3.36}{2022/02/03}{using \file{scrlogo.dtx} to define the logo}
%
% \GetFileInfo{scrlfile-hook.dtx}
% \title{\pkg{scrlfile} \KOMAScript{} Sub-Packages \pkg{scrlfile-hook} and
% \pkg{scrlfile-hook-3.34}}
% \author{\href{mailto:komascript@gmx.info}{Markus Kohm}}
% \date{Revision \fileversion{} of \filedate}
% \maketitle
% \begin{abstract}
%   This package provides hooks before and after loading files, packages or
%   classes. It also provides a hook after the last \cs{clearpage} of the
%   document. It allowes to replace files, packages and classes by other
%   files, packages and classes. It is inteded to be used by package and class
%   authors but may also be used by \LaTeX{} users.
% \end{abstract}
% \tableofcontents
% 
%
% \section{User Manual}
%
% \pkg{scrlfile-hook} implements the \LaTeX-hook-based part of
% \pkg{scrlfile}. \pkg{scrlfile-hook-3.34} is a variant for \LaTeX{} kernel
% versions before 2021/06/01.
%
% There isn't any user manual for the user level \LaTeXe{} commands in this
% file. Please see the manual of \pkg{scrlfile} for more information about
% \pkg{scrlfile-hook} and \pkg{scrlfile-hook-3.34}.
%
% This section, however, contains the user manual of the \LaTeX3{} package
% author commands.
%
% \begin{function}[TF, added = 2020-08-26]
%   {\scrlfile_if_class_loaded:n,\scrlfile_if_package_loaded:n}
%   \begin{syntax}
%     \cs{scrlfile_if_class_loaded:nTF} \Arg{class name} \Arg{true code} \Arg{false code}
%     \cs{scrlfile_if_package_loaded:nTF} \Arg{package name} \Arg{true code} \Arg{false code}
%   \end{syntax}
%   Tests if the class \meta{class name} resp. the package \meta{package name}
%   has been loaded completely. It runs the \Arg{true code} only, if the input
%   of the class file with the name \meta{class
%   name}\file{.}\cs{@clsextension} resp. the package file with the name 
%   \meta{package name}\file{.}\cs{@pkgextension} has already been
%   finished. It runs the \Arg{false code}, if the class or package has not
%   been loaded or the input of the class or package file is still in
%   progress.
% \end{function}
%
%
% \MaybeStop{\PrintIndex}
%
% \section{Implementation of \pkg{scrlfile-hook}}
%
%    \begin{macrocode}
%<@@=scrlfile>
%    \end{macrocode}
%
% Test whether the used \LaTeX{} provides all commands we need.
% \changes{v3.34}{2021/04/21}{make it more robust agains not recommended
%   direct usage}
% \changes{v3.35}{2021/10/31}{new file version because \LaTeX{} kernel
%   2021/11/15 has changed the field order of gerneric hooks}
% \changes{v3.36}{2022/02/03}{package name in messages fixed}
%    \begin{macrocode}
\@ifundefined{IfFormatAtLeastTF}{%
  \PackageError{scrlfile-hook}{not recommended usage of package}{%
    It seems this package has been loaded directly using a LaTeX
    version\MessageBreak
    prior to 2020-10-01. This is not recommended. Please always load
    package\MessageBreak
    scrlfile instead of scrlfile-hook.\MessageBreak
    If you would continue, I will try to load scrlfile-patcholdlatex%
  }%
  \RequirePackage{scrlfile-patcholdlatex}\endinput
}{%
  \IfFormatAtLeastTF{2020/10/01}{%
    \IfFormatAtLeastTF{2021/11/15}{%
%<*3.34>
      \PackageError{scrlfile-hook-3.34}{LaTeX too new for this package}{%
        It seems this package has ben loaded directly using a LaTeX
        version\MessageBreak
        newer than 2021-06-01. This is not recommended. Please always load
        package\MessageBreak
        scrlfile instead of scrlfile-hook-3.34.\MessageBreak
        If you would continue, I will try to load scrlfile-hook%
      }%
      \RequirePackage{scrlfile-hook}\endinput
%</3.34>
    }{%
%<*current>
      \PackageError{scrlfile-hook}{LaTeX too old for this package}{%
        It seems this package has been loaded directly using LaTeX
        version\MessageBreak
        prior to 2021-11-15. This is not recommended. Please always load
        package\MessageBreak
        scrlfile instead of scrlfile-hook.\MessageBreak
        If you would continue, I will try to load scrlfile-hook-3.34%
      }%
      \RequirePackage{scrlfile-hook-3.34}\endinput
%</current>
    }%
  }{%
    \PackageError{scrlfile-hook}{LaTeX too old for this package}{%
      It seems this package has been loaded directly using a LaTeX
      version\MessageBreak
      prior to 2020-10-01. This is not recommended. Please always load
      package\MessageBreak
      scrlfile instead of scrlfile-hook.\MessageBreak
      If you would continue, I will try to load scrlfile-patcholdlatex%
    }%
    \RequirePackage{scrlfile-patcholdlatex}\endinput
  }%
}
%    \end{macrocode}
%
%
% \subsection{Before and After Commands}
%
% The hook implementation is based on a \LaTeX{} version that provides
% \LaTeX3. So it makes sense to use it.
%
%    \begin{macrocode}
\ExplSyntaxOn
%    \end{macrocode}
%
% \begin{macro}[updated = 2021-10-31]{\BeforeFile}
% The hook version of this command is just a wrapper to the corresponding
% \LaTeX{} file hooks. It supports a mandatory \meta{file} argument, an
% optional \meta{label} argument and a mandatory \meta{hook code} argument.
% From \LaTeX{} kernel version 2021/11/15 the hook has
% been changed from \texttt{file/before/\meta{file name}} to
% \texttt{file/\meta{file name}/before}.
%    \begin{macrocode}
\NewDocumentCommand \BeforeFile { m }
  {
%<3.34>    \AddToHook { file / before / #1 } 
%<current>    \AddToHook { file / #1 / before } 
  }
%    \end{macrocode}
% \end{macro}
%
% \begin{macro}[updated = 2021-10-31]{\AfterFile}
% The hook version of this command is just a wrapper to the corresponding
% \LaTeX{} file hooks. It supports a mandatory \meta{file} argument, an
% optional \meta{label} argument and a mandatory \meta{hook code} argument.
% From \LaTeX{} kernel version 2021/11/15 the hook has
% been changed from \texttt{file/after/\meta{file name}} to
% \texttt{file/\meta{file name}/after}.
%    \begin{macrocode}
\NewDocumentCommand \AfterFile { m }
  {
%<3.34>    \AddToHook { file / after / #1 } 
%<current>    \AddToHook { file / #1 / after } 
  }
%    \end{macrocode}
% \end{macro}
%
% \begin{macro}{\BeforeClass,\BeforePackage}
% The hook version of these commands are also wrappers to the
% \texttt{file/before} hooks, because the \meta{code} should also be executed
% already in the class or package context.
%    \begin{macrocode}
\NewDocumentCommand \BeforeClass { m }
  {
    \BeforeFile { #1.\@clsextension }
  }
\NewDocumentCommand \BeforePackage { m }
  {
    \BeforeFile { #1.\@pkgextension }
  }
%    \end{macrocode}
% \end{macro}
%
% \begin{macro}[updated = 2021-10-31, updated = 2021-11-09]
%   {\AfterAtEndOfClass,\AfterAtEndOfPackage}
% \changes{v3.35}{2021/10/31}{field order of generic hooks changed in
%   \LaTeX{} 2021/11/15}
% \changes{v3.35}{2021/11/09}{hooks became one-time hooks in
%   \LaTeX{} 2021/11/15}
% With version 3.32 the syntax of these commands have been changed. Now, there
% is also a star variant, that runs the \meta{code} immediately if the class
% or package has already been loaded completely. Otherwise and in the normal
% variant the \meta{code} is added to the \texttt{class/after} or
% \texttt{package/after} hook, because this hook is used outside the context
% of the class or package and after the \cs{AtEndOfClass} or
% \cs{AtEndOfPackage} code.
% From \LaTeX{} kernel version 2021/11/15 the hooks has
% been changed from \texttt{class/after/\meta{class name}} to
% \texttt{class/\meta{class name}/after} and from
% \texttt{package/after/\meta{package name}} to
% \texttt{package/\meta{package name}/after}. Additionally the hooks became
% one-time hooks. So they are executed immediately, if the class or package
% has already been loaded. Because of this, we additionally have to test this
% case in the non star variant.
%    \begin{macrocode}
\NewDocumentCommand \AfterAtEndOfClass { s m o +m }
  {
    \IfBooleanTF { #1 }
      {
        \scrlfile_if_class_loaded:nTF { #2 }
          { #4 }
%<3.34>          { \hook_gput_code:nnn { class / after / #2 } { #3 } { #4 } }
%<current>          { \hook_gput_code:nnn { class / #2 / after } { #3 } { #4 } }
      }
%<3.34>      { \hook_gput_code:nnn { class / after / #2 } { #3 } { #4 } }
%<*current>
      {
        \scrlfile_if_class_loaded:nF
          { #2 }
          { \hook_gput_code:nnn { class / #2 / after } { #3 } { #4 } }
      }
%</current>
  }
\NewDocumentCommand \AfterAtEndOfPackage { s m o +m }
  {
    \IfBooleanTF { #1 }
      {
        \scrlfile_if_package_loaded:nTF { #2 }
          { #4 }
%<3.34>          { \hook_gput_code:nnn { package / after / #2 } { #3 } { #4 } }
%<current>          { \hook_gput_code:nnn { package / #2 / after } { #3 } { #4 } }
      }
%<3.34>      { \hook_gput_code:nnn { package / after / #2 } { #3 } { #4 } }
%<*current>
      {
        \scrlfile_if_package_loaded:nF
          { #2 }
          { \hook_gput_code:nnn { package / #2 / after } { #3 } { #4 } }
      }
%</current>
  }
%    \end{macrocode}
% \end{macro}
%
% \begin{variable}[added = 2020-06-26,
%   updated = 2021-01-01]{\g__scrlfile_input_file_seq}
% \changes{3.33}{2021/01/01}{message definition moved outside function}
% \cs{g__scrlfile_input_file_seq} is a sequence of active file inputs (without
% path information). Two global hooks are used to setup the sequence. The new
% conditional tests if a file is in the sequence. And a message is defined for
% the case of more pops than pushs to the stack.
%    \begin{macrocode}
\seq_new:N \g__scrlfile_input_file_seq
\hook_gput_code:nnn { file / before } { . }
  { \seq_gpush:Nx \g__scrlfile_input_file_seq { \CurrentFile }  }
\msg_new:nnn { scrlfile-hook } { too-many-pops }
  {
    More~file~names~popped~from~stack~than~put~to.~
    This~should~never~happen.~
    However,~it~could~happen~if~scrlfile-hook~is~loaded~by~another~
    package~or~class.~In~this~case~some~packages~or~classes~are~not~
    recognised~correctly.
  }
\hook_gput_code:nnn { file / after } { . }
  {
    \seq_gpop:NNF \g__scrlfile_input_file_seq \l_tmpa_seq
      {
        \msg_warning:nn { scrlfile-hook } { too-many-pops }
      }
  }
%    \end{macrocode}
% Unfortunately we need an ugly hack to initialise the stack using an internal
% kernel variable. This is a no go but I do not know a better solution for
% this, because loading of the package could be done late.
% \begin{description}
% \item[ToDo:] Decide, if the second or fourth token is correct.  If fourth,
%   \cs{CurrentFile} has to be used always instead of \cs{CurrentFileUsed}.
% \end{description}
%    \begin{macrocode}
\cs_if_exist:NTF \g__filehook_input_file_seq
  {
    \seq_map_inline:Nn \g__filehook_input_file_seq
      {
        \seq_gput_right:Nx \g__scrlfile_input_file_seq
          { \tl_item:nn { #1 } { 2 } }
      }
  }
  {
    \seq_gpush:Nx \g__scrlfile_input_file_seq { }
    \cs_if_exist:NTF \CurrentFileUsed
      { \seq_gpush:Nx \g__scrlfile_input_file_seq { \CurrentFileUsed } }
      { \seq_gpush:Nx \g__scrlfile_input_file_seq { \CurrentFile } }
  }
%    \end{macrocode}
% \end{variable}
% \begin{function}[TF, added = 2020-06-26]{\@@_if_loading:n}
% Test if the file name is in the file name list.
%    \begin{macrocode}
\prg_new_protected_conditional:Npnn \__scrlfile_if_loading:n #1 { T, F, TF }
  {
    \str_set:Nx \l_tmpa_str { #1 }
    \seq_if_in:NxTF \g__scrlfile_input_file_seq { \str_use:N \l_tmpa_str }
      { \prg_return_true: }
      { \prg_return_false: }
  }
%    \end{macrocode}
% \end{function}
%
% \begin{function}[TF, added = 2020/08/22]
%   {\scrlfile_if_class_loaded:n,\scrlfile_if_package_loaded:n}
% \cs{scrlfile_if_class_loaded:nTF} is similar to \cs{@ifclassloaded} and
% \cs{scrlfile_if_package_loaded:nTF} is similar to \cs{@ifpackageloaded} but in
% opposite to those they test, if the class or package has been loaded
% completely.
%    \begin{macrocode}
\prg_new_protected_conditional:Npnn \scrlfile_if_class_loaded:n #1 { T, F, TF }
  {
    \@ifclassloaded { #1 }
      {
        \__scrlfile_if_loading:nTF { #1.\@clsextension }
          { \prg_return_false: }
          { \prg_return_true: }
      }
      {
        \prg_return_false:
      }
  }
\prg_new_protected_conditional:Npnn \scrlfile_if_package_loaded:n #1 { T, F, TF }
  {
    \@ifpackageloaded { #1 }
      {
        \__scrlfile_if_loading:nTF { #1.\@pkgextension }
          { \prg_return_false: }
          { \prg_return_true: }
      }
      {
        \prg_return_false:
      }
  }
%    \end{macrocode}
% \end{function}
%
% \begin{macro}[updated = 2021-10-31]{\AfterClass,\AfterPackage}
% With version 3.32 these do not support plus or exclamation mark variants,
% but only the normal and the star variants. Instead of the plus or
% exclamation mark variants users should use the star variant of
% \cs{AfterAtEndOfClass} and \cs{AfterAtEndOfPackage}. The commands use the
% \texttt{file/after} hook, because the user manual declares, that \meta{code}
% is used before the code of \cs{AtEndOfClass} or \cs{AtEndOfPackage}.
% From \LaTeX{} kernel version 2021/11/15 the hook has
% been changed from \texttt{file/after/\meta{file name}} to
% \texttt{file/\meta{file name}/after}.
%    \begin{macrocode}
\NewDocumentCommand \scrlfile@AfterClass { s m o +m }
  {
    \IfBooleanTF { #1 }
      {
        \@ifclassloaded{ #2 }
          { #4 }
          {
            \hook_gput_code:nnn
%<3.34>              { file / after / #2.\@clsextension }
%<current>              { file / #2.\@clsextension / after }
              { #3 }
              { #4 }
          }
      }
      {
%<3.34>        \hook_gput_code:nnn { file / after / #2.\@clsextension } { #3 } { #4 }
%<current>        \hook_gput_code:nnn { file / #2.\@clsextension / after } { #3 } { #4 }
      }
  }
\NewDocumentCommand \AfterClass { } { \scrlfile@AfterClass }
\NewDocumentCommand \scrlfile@AfterPackage { s m o +m }
  {
    \IfBooleanTF { #1 }
      {
        \@ifpackageloaded{ #2 }
          { #4 }
          {
            \hook_gput_code:nnn
%<3.34>              { file / after / #2.\@pkgextension }
%<current>              { file / #2.\@pkgextension / after }
              { #3 }
              { #4 }
          }
      }
      {
%<3.34>        \hook_gput_code:nnn { file / after / #2.\@pkgextension } { #3 } { #4 }
%<current>        \hook_gput_code:nnn { file / #2.\@pkgextension / after } { #3 } { #4 }
      }
  }
\NewDocumentCommand \AfterPackage { } { \scrlfile@AfterPackage }
%    \end{macrocode}
% \end{macro}
%
%
% \subsection{File Substitution}
%
% With new file hooks the substitutions are so easy we even do not need
% \LaTeX3 syntax to implement them.
%
% \begin{macro}{\ReplaceInput}
% This is only the simplest wrapper to \cs{declare@file@substitution}.
%    \begin{macrocode}
\NewDocumentCommand{\ReplaceInput}{}{\declare@file@substitution}
%    \end{macrocode}
% \end{macro}
%
% \begin{macro}{\ReplaceClass}
% \begin{macro}{\ReplacePackage}
% These are also wrappers to \cs{declare@file@substitution}. But in this case
% we also have to add the extension.
%    \begin{macrocode}
\NewDocumentCommand\ReplaceClass{mm}{%
  \declare@file@substitution{#1.\@clsextension}{#2.\@clsextension}%
}
\NewDocumentCommand\ReplacePackage{mm}{%
  \declare@file@substitution{#1.\@pkgextension}{#2.\@pkgextension}%
}
%    \end{macrocode}
% \end{macro}
% \end{macro}
%
% \begin{macro}{\UnReplaceInput}
% This is the simplest wrapper to \cs{undeclare@file@substitution}.
%    \begin{macrocode}
\NewDocumentCommand{\UnReplaceInput}{}{\undeclare@file@substitution}
%    \end{macrocode}
% \end{macro}
%
% \begin{macro}{\UnReplaceClass}
% \begin{macro}{\UnReplacePackage}
% Also very simple but again we have to add the extension.
%    \begin{macrocode}
\NewDocumentCommand\UnReplaceClass{m}{%
  \undeclare@file@substitution{#1.\@clsextension}%
}
\NewDocumentCommand\UnReplacePackage{mm}{%
  \undeclare@file@substitution{#1.\@pkgextension}%
}
%    \end{macrocode}
% \end{macro}
% \end{macro}
%
% \subsection{Prevent Package from Loading}
%
% To store and reset the hole list of prevents we use a comma separated
% list. So we need again use \LaTeX3.
%
% \begin{variable}[added = 2020-09-02]{\g__scrlfile_prevents_clist}
% This local variable stores the list of files, that should be prevented from
% loading. It is needed to stay compatible with old \file{scrlfile}. Without
% this compatibility purpose we would be able to use simple wrappers to
% \cs{disable@package@load} and \cs{reenable@package@load}.
%    \begin{macrocode}
\clist_new:N \g__scrlfile_prevent_clist
%    \end{macrocode}
% \end{variable}
%
% \begin{macro}{\PreventPackageFromLoading}
% This is more than a wrapper to \cs{disable@package@load} because we have to
% manage an internal list and it is documented, that the re-enabling does not
% undefine the \meta{alternate-code} setting but only disables it. So we have
% to extra store it. Note: Local loading of packages does not make sense, so
% local changes of the prevent list also does not make sense. Therefore this
% is a global acting command!
%    \begin{macrocode}
\NewDocumentCommand \PreventPackageFromLoading { s +o m }
  {
    \clist_set:Nx \l__scrlfile_package_clist { #3 }
    \clist_map_inline:Nn \l__scrlfile_package_clist
      {
        \@ifpackageloaded { ##1 }
          {
            \IfBooleanTF { #1 } { \msg_info:nnn } { \msg_warning:nnn }
              { scrlfile } { no-prevent-for-already-loaded } { ##1 }
          }
          {
            \clist_if_in:NnF \g__scrlfile_prevent_clist { ##1 }
              { \clist_gput_right:Nn \g__scrlfile_prevent_clist { ##1 } }
            \tl_if_exist:cF { g__scrlfile_exclude_package_##1_tl }
              {
                \tl_new:c { g__scrlfile_exclude_package_##1_tl }
              }
            \IfValueT { #2 }
              {
                \tl_gput_right:cn { g__scrlfile_exclude_package_##1_tl } { #2 }
              }
            \disable@package@load { ##1 }
              { \tl_use:c { g__scrlfile_exclude_package_##1_tl } }
          }
      }
    \clist_clear:N \l__scrlfile_package_clist  
  }
%    \end{macrocode}
%
% \begin{macro}{\l__scrlfile_package_clist}
% One local variable is used to process the \meta{package-list} argument of
% \cs{PreventPackageFromLoading}.
%    \begin{macrocode}
\clist_new:N \l__scrlfile_package_clist
%    \end{macrocode}
% \end{macro}
%
% And here comes the message, that could be used either as a warning or as an
% info.
%    \begin{macrocode}
\msg_new:nnn { scrlfile } { no-prevent-for-already-loaded }
  {
    Cannot~prevent~package~`#1'~from~being~loaded,~
    because~it~has~been~loaded~already~before~line~\msg_line_number:
  }
%    \end{macrocode}
% \end{macro}
%
% \begin{macro}{\StorePreventPackageFromLoading}
% This simply copies the internal \texttt{clist} to a macro.
%    \begin{macrocode}
\NewDocumentCommand \StorePreventPackageFromLoading { m }
  { \edef #1 { \clist_use:Nn \g__scrlfile_prevent_clist { , } } }
%    \end{macrocode}
% \end{macro}
%
% \begin{macro}{\ResetPreventPackageFromLoading}
% Map the internal list to a function that re-enables the package. At the end
% the internal list is cleared.
%    \begin{macrocode}
\NewDocumentCommand \ResetPreventPackageFromLoading {}
  {
    \clist_map_function:NN \g__scrlfile_prevent_clist \reenable@package@load
    \clist_gclear:N \g__scrlfile_prevent_clist
  }
%    \end{macrocode}
% \end{macro}
%
% \begin{macro}{\UnPreventPackageFromLoading}
% Here again the argument is a \meta{package-list} not only one
% \meta{package}. So we have to build a local \texttt{clist} and walk through
% it.
%    \begin{macrocode}
\NewDocumentCommand \UnPreventPackageFromLoading { s m }
  {
    \clist_set:Nx \l__scrlfile_package_clist { #2 }
    \clist_map_inline:Nn \l__scrlfile_package_clist
      {
        \clist_if_in:NnT \g__scrlfile_prevent_clist { ##1 }
          {
            \clist_gremove_all:Nn \g__scrlfile_prevent_clist { ##1 }
            \reenable@package@load { ##1 }
            \IfBooleanT { #1 }
              { \cs_undefine:c { g__scrlfile_exclude_package_##1_tl } }
          }
      }
  }
%    \end{macrocode}
% \end{macro}
%
%
% \subsection{Extra Document Hooks}
%
% \begin{macro}{\BeforeClosingMainAux}
% \changes{v3.32}{2020/09/10}{optional argument added}
% Here we cannot simply wrap this to the hook
% \texttt{enddocument/afterlastpage}, because it is documented, that inside
% the \meta{code} \cs{protected@write} is replaced by
% \cs{immediate@protected@write}. So we have to take extra care to it.
%    \begin{macrocode}
\NewDocumentCommand \BeforeClosingMainAux { o m }
  {
    \hook_gput_code:nnn { enddocument / afterlastpage } { #1 }
      {
        \debug_suspend:
        \RenewDocumentCommand \BeforeClosingMainAux { m } { ##1 }
        \cs_set_eq:NN \__scrlfile_protected@write:Nnn \protected@write
        \cs_set_eq:NN \protected@write \protected@immediate@write
        #2
        \cs_set_eq:NN \protected@write \__scrlfile_protected@write:Nnn
        \debug_resume:
      }
  }  
%    \end{macrocode}
% \end{macro}
%
% \begin{macro}{\AfterReadingMainAux}
% \changes{v3.32}{2020/09/10}{optional argument added}
% Here we the exact same problem as with \cs{BeforeClosingMainAux}.
%    \begin{macrocode}
\NewDocumentCommand \AfterReadingMainAux { o m }
  {
    \hook_gput_code:nnn { enddocument / afteraux } { #1 }
      {
        \debug_suspend:
        \RenewDocumentCommand \AfterReadingMainAux { m } { ##1 }
        \cs_set_eq:NN \__scrlfile_protected@write:Nnn \protected@write
        \cs_set_eq:NN \protected@write \protected@immediate@write
        #2
        \cs_set_eq:NN \protected@write \__scrlfile_protected@write:Nnn
        \debug_resume:
      }
  }  
%    \end{macrocode}
% \end{macro}
%
%    \begin{macrocode}
\ExplSyntaxOff
%    \end{macrocode}
%
% \subsection{Kernel Extensions not Using \LaTeX3}
%
% \begin{macro}{\protected@immediate@write}
% Like \LaTeX{} kernel's |\protected@write| but using |\immediate\write|. In
% this case it is even not a good idea to protect |\thepage|!
%    \begin{macrocode}
\ProvideDocumentCommand\protected@immediate@write{m+m+m}
  {%
    \begingroup
      #2%
      \let\protect\@unexpandable@protect
      \edef\reserved@a{\immediate\write#1{#3}}%
      \reserved@a
    \endgroup
    \if@nobreak\ifvmode\nobreak\fi\fi
  }
%    \end{macrocode}
% \end{macro}
% 
% \Finale
% \PrintChanges
%
% \endinput
% Local Variables:
% mode: doctex
% ispell-local-dictionary: "en_US"
% eval: (flyspell-mode 1)
% TeX-master: t
% TeX-engine: luatex-dev
% eval: (setcar (or (cl-member "Index" (setq-local TeX-command-list (copy-alist TeX-command-list)) :key #'car :test #'string-equal) (setq-local TeX-command-list (cons nil TeX-command-list))) '("Index" "mkindex %s" TeX-run-index nil t :help "makeindex for dtx"))
% End:
